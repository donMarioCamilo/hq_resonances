\subsection{Preliminaries}\label{preliminaries}
\paragraph{Born - Oppenheimer approximation in atomic/molecular physics:}
When dealing with a system consisting of many electrons and nuclei, the starting Hamiltonian involves too many coupled degrees of freedom. In the first place, by knowing that the nuclei are much heavier than the electrons, one can assume that the former move more slowly than the latter. Thus, it is reasonable to assume a Hamiltonian for N electrons in the field generated by a static configuration of K nuclei (\ref{eq:BO_hamiltonian}), and therefore to consider the nuclear-nuclear repulsion to be constant for a given electronic configuration. This is the widely known Born-Oppenheimer approximation \cite{sutcliffe}. %TODO Check: Moreover, the kinetic energy of the nuclei can be neglected, such that the total energy of the system is given by the sum of the energy of the electrons and the electrostatic energy of the nuclei \cite{computationalphysics}.
With the Hartree-Fock procedure the electronic energy will be minimised. To find the ground state of the whole system and obtaining an \emph{adiabatic} potential for the electrons, the procedure must be repeated changing the positions of the nuclei.%TODO Check this last phrase
\begin{equation}
    H_{BO} =\sum_{i=1}^N\frac{p_i^2}{2}+
    \frac{1}{2} \sum_{i,j=1;i\neq j}^N \frac{1}{\left|\mathbf{r}_i -\mathbf{r}_j\right|} - 
    \sum_{n=1}^K\sum_{i=1}^N \frac{Z_n}{\left|\mathbf{r}_i -\mathbf{R}_n\right|}
    \label{eq:BO_hamiltonian}
\end{equation}
where atomic units have been used \cite{szabo}. It should be noted that the Born-Oppenheimer Hamiltonian omits some interactions such as spin-orbit and that it neglects quantum fluctuations.
\paragraph{Independent particle method:}
The interelectronic repulsion terms in (\ref{eq:BO_hamiltonian}) make impossible to find an eigenfunction of the Hamiltonian, since it is not separable. The main idea for solving this problem is by decoupling its terms with the independent particle method. This is, by considering that the interaction between an electron and the remaining ones can be incorporated in the Hamiltonian as if the remaining ones created an averaged electrostatic field on the electron. In this way, an uncoupled, effective Hamiltonian is obtained for each electron:
\begin{equation}
    H_{IP}=\sum_{i=1}^N\left[\frac{p_i^2}{2}+V(\mathbf{r_i})\right]=\sum_{i=1}^N h(i)
    \label{eq:IP_hamiltonian}
\end{equation}
where $V(\mathbf{r_i})$ is a potential dependent on the chosen configuration of the nuclei, usually a non-local operator. This independent particle Hamiltonian will be further discussed later on.\\\\
The fact that $H_{IP}$ is a sum of one-electron Hamiltonians makes it possible to find an eigenfunction as a product of spin orbital wave functions:
\begin{equation}
	\Psi (q_1,...,q_N) = \psi_1(q_1)... \psi_N(q_N)
	\label{eq:prod_spinorbitals}
\end{equation}
where $q = (\mathbf{r},s)$ contains the spatial and spin coordinates of one electron. This solution wave function may also be referred as a \emph{Hartree product}. The corresponding eigenvalue $E$ is then a sum of the spin orbital energies corresponding to each $\psi_k$: $E = \sum_{k=1}^N\epsilon_k$. 

It should be noted that a solution wave function of this form has an uncorrelated probability distribution, since the probability density of finding each particle $i$ in the volume element $d\mathbf{x}_i$, centred at $\mathbf{x}_i$, can be written as a product of the one-particle probability densities:
\begin{equation}
	\rho (q_1,...,q_N) = \left|\psi_1(q_1)\right|^2... \left|\psi_N(q_N)\right|^2
	\label{eq:prob_density}
\end{equation}
Therefore, a solution of this form distinguishes particles by the spin orbital they occupy. Nevertheless, for satisfying the antisymmetry principle, indistinguishability of electrons must be taken into account with an antisymmetric wave function. Hartree-Fock theory will fix this deficit.

\paragraph{Self-consistent-field (SCF)} Lets now search for an approximate independent particle Hamiltonian for the case of K nuclei and N electrons within the Born-Oppenheimer aproximation. For this introduction the antisymmetry requirement will not be taken into account, as the solution wave function will be chosen to have the form of a Hartree product.\\\\
In first place, the spin-independent Hamiltonian acting on the solution wave function reads:%TODO Jupyter_Levine citation? Hartree scf is not  directly used on spin-orbitals :(
\begin{equation}
    \left[-\sum_{i=1}^N\frac{1}{2}\nabla_i^2+
    \sum_{i,j=1;i\neq j}^N \frac{1}{\left|\mathbf{r}_i -\mathbf{r}_j\right|} - 
    \sum_{n=1}^K\sum_{i=1}^N \frac{Z_n}{\left|\mathbf{r}_i -\mathbf{R}_n\right|}\right]\Psi (q_1,...,q_N) = E \Psi (q_1,...,q_N)
    \label{eq:hart1}
\end{equation}
For separating this equation, such that an independent particle equation is obtained for each spin-orbital $\psi_k$, both sides of (\ref{eq:hart1}) can be multiplied from the left by $N-1$ spin orbitals\footnote{This is, multiplying by $\prod_{m\neq k}^N\psi_m^* (q_m)$.} $\psi_m^* (q_m)$ and integrated over every $q_m$, therefore arriving to%TODO Check N-1 or N
\begin{equation}
    \left[-\frac{1}{2}\nabla^2 - 
    \sum_{n=1}^K \frac{Z_n}{\left|\mathbf{r} -\mathbf{R}_n\right|} + 
    \sum_{l=1}^N \int{dq'\left|\psi_l(q')\right|^2\frac{1}{\left|\mathbf{r} -\mathbf{r'}\right|}}\right]\psi_k (q) = E' \psi_k (q)
    \label{eq:hart2}
\end{equation}
where normalisation of the spin orbitals is assumed and several constants have been absorbed into the new variable $E'$. The third term on the left hand side will be referenced later on as the Hartree potential, which represents the Coulomb energy of the $k$-th electron in the average field generated by the charge distribution caused by all the N electrons\footnote{A usual nuisance is the self-coupling of the $k$-th electron. Hartree-Fock theory will solve this problem by taking into account the antisymmetry requirement of the wave function.}.

One can see that the Hamiltonian acting on $\psi_k$ in this equation has the form of an effective one-electron Hamiltonian $h(i)$. In this way, an independent particle equation (\ref{eq:IP_hamiltonian}) has been obtained. %In addition, $V(\mathbf{x})$ can be identified as a non-local operator, since it depends on the unknown solution wave function.\\\\%It is worth noting that this could be done because the wave function is a product of spin orbitals.
%TODO Thanks to prod of spin-orb
%TODO Non-local operator?
For finding the spin-orbitals the \textbf{Hartree self-consistent-field method} shall be used \cite{Blinder}. For doing so, a trial ground state solution $\psi^{(0)}$ is proposed for constructing the potential. Then, solving the spin orbital equations (\ref{eq:hart2}) yields an improved ground state $\psi^{(1)}$, which is then used to build a new potential. This procedure is repeated until convergence is met\footnote{Different criteria may be used for checking convergence. For example, the ground state not deviating appreciably between iterations. This will be further discussed when the Hartree-Fock computer program is presented.}.

\paragraph{Antisymmetry and correlation} %For finding a solution wave function that can satisfy the antisymmetry principle it is fundamental to remember that electrons are identical particles. This means that the Hamiltonian commutes with the particle-exchange operator

Any valid wave function must be antisymmetric with respect to particle exchange. To extend the Hartree product of equation (\ref{eq:prod_spinorbitals}) to satisfy the antisymmetry principle, an antisymmetric linear combination of the spin orbitals can be used. A usual such ansatz is the Slater determinant
\begin{equation}
	\Psi (q_1,...,q_N)= \frac{1}{\sqrt{N!}}
	\begin{vmatrix}
  \psi_1 (q_1) & \psi_2 (q_1) & \cdots & \psi_N (q_1)\\
  \psi_1 (q_2) & \psi_2 (q_2) & \cdots & \psi_N (q_2)\\
  \vdots & \vdots & \ddots & \vdots \\
  \psi_1 (q_N) & \psi_2 (q_N) & \cdots & \psi_N (q_N)
\end{vmatrix}
\end{equation}
From now on, a more convenient, simplified notation for a normalized Slater determinant will be used \cite{szabo}:
\begin{equation}
		\Psi (q_1,...,q_N) \equiv \ket{\psi_1(q_1)... \psi_N(q_N)}\equiv\ket{\psi_1... \psi_N}
\end{equation}
It should be noted that antisymmetrisation introduces correlation effects, which can be seen by considering the probability density of finding two electrons with coordinates $q_1$ and $q_2$:
\begin{equation}
	\rho(q_1, q_2) = \frac{1}{N(N-1)}	\sum_{k,l}\left[\left|\psi_k(q_1)\right|^2\left|	\psi_l(q_2)\right|^2-\psi_k^*(q_1) 	\psi_k(q_2) \psi_l(q_1) \psi_l^*(q_2)\right]
	\label{eq:correlated_prob}
\end{equation}
%Lets consider spin orbitals that can be written as a product of a spatial orbital and a one-particle spin wave function, i.e. $\psi_k(\mathbf{x}_k) = \chi_k(\mathbf{r}_k)\alpha(\omega_k)$. 
From this it can be deduced that correlation exchange is only present for equal spin orbitals, whereas opposite spin orbitals remain uncorrelated (as they do in equation (\ref{eq:prob_density})). Moreover, if electrons have parallel spin and $\mathbf{r}_1= \mathbf{r}_2$, then $\rho(q_1, q_1)=0$, thus satisfying Pauli's exclusion principle. %TODO ref.
For this it is said that the electrons are surrounded by an \emph{exchange} or \emph{Fermi hole}, in which other electrons with the same spin are hardly found \cite{szabo,computationalphysics,slater1}.
%TODO Want to justify it or just reference computational? If just reference, then do the same when explaining non-correlation. Review scf. If only the citation, explain qualitatively (szabo)
%%%%%%%%%%%%%%%%%%%%%%%%%%%%%%%%%%%%%%%%%%%%%%%%%%%%%%%%%%%%%%%