%This subsection used to be between HF theory and HF computer program
\subsection{Basis functions and the Roothaan equations}
Roothan introduced the idea of choosing the set of solutions ---the Hartree-Fock orbitals $\psi_k$--- as linear combinations of a complete set of known \textbf{basis functions} $\chi_p$ \cite{roothaan}. This is,
\begin{equation}
	\psi_k(\mathbf{x}) = \sum_{q=1}^M C_{qk}\chi_q(\mathbf{x})
	\label{eq:basis_functions}
\end{equation}
where the expansion coefficients $C_{qk}$ are found by the SCF procedure.

For RHF calculations, only the orbital part $\chi_q (\mathbf{r}) $ of the parametrisation is used, since spin dependence has been summed over.\\\\
For computational methods it is proper to find the matrix form of the Hartree-Fock equations. By multiplying (\ref{eq:new_HF}) with $\chi_p$ from the left the \textbf{Roothaan equations} are obtained:
\begin{equation}
	\mathbf{F}\mathbf{C}_k = \varepsilon_k\mathbf{S}\mathbf{C}_k
	\label{eq:roothaan_equation}
\end{equation}
where $\mathbf{S}$ is the overlap matrix, which elements are defined as
\begin{equation}
	S_{pq}=\ip{\chi_p}{\chi_q}
\end{equation}
and the Fock matrix has the form
\begin{equation}
	F_{pq} = h_{pq}+\frac{1}{2}\sum_{r,s}P_{rs}\left[2\mel{pr}{g}{qs}-\mel{pr}{g}{sq}\right]
	\label{eq:fock_matrix}
\end{equation}
Note that $p,q,r$ and $s$ are indexes that reference the basis functions. In equation (\ref{eq:fock_matrix}) various matrix elements are introduced. These are:
\begin{itemize}
\item For the uncoupled one-electron hamiltonian:
\begin{equation}
	h_{pq}=\mel{p}{h}{q}=\int{d^3r \chi_p^*(\mathbf{r})\left[-\frac{1}{2}\nabla^2-\sum_n\frac{Z_n}{\left|\mathbf{R}_n-\mathbf{r}\right|}\right]\chi_q(\mathbf{r})}
	\label{eq:h_matrix}
\end{equation}

\item For the interelectronic terms, the 4-dimensional matrix $g$ elements are introduced as two-electron integrals:
\begin{equation}
	\mel{pr}{g}{qs}=\int{d^3r_1 d^3r_2 \chi_p^*(\mathbf{r}_1)\chi_r^*(\mathbf{r}_2)\frac{1}{\left|\mathbf{r}_1-\mathbf{r}_2\right|}\chi_q(\mathbf{r}_1)\chi_s(\mathbf{r}_2)}
	\label{eq:g_matrix}
\end{equation}

\item The density matrix $\mathbf{P}$ for RHF, which, analogously to the definition of the density matrix in quantum mechanics \cite{densitymatrix}:
\begin{equation}
	P_{pq} = 2\sum_{k}C_{pk}C_{qk}^*
	\label{eq:density_matrix}
\end{equation}
\end{itemize}
For solving the Roothaan equations (\ref{eq:roothaan_equation}) the SCF method is implemented as follows: by making a first guess for the density matrix (this is, making a guess for the wave function expansion coefficients), one can compute the Coulomb and exchange contributions. By doing so, the Fock matrix is easily calculated, thus arriving to the Roothaan equation. Then the generalised eigenvalue problem is solved, yielding as eigenvectors the new expansion coefficients. These are used for building a new density matrix, which is then used to repeat the procedure. When convergence is met, the ground state energy can be easily calculated. A more specific approach to the SCF will be presented in the following section.%, where the Hartree-Fock computer program is presented.
